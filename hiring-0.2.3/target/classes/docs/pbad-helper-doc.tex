\documentclass{article}

\usepackage[brazil]{babel}
\usepackage[T1]{fontenc}
\usepackage[a4paper]{geometry}
\usepackage[colorlinks, urlcolor=gray, citecolor=gray, linkcolor=gray]{hyperref}
\usepackage[utf8]{inputenc}
\usepackage{parskip, tikz}

\title{Seleção PBAD/LabSEC - Etapa final}
\author{}
\date{\today}

\begin{document}

\maketitle

\section{Introdução}\label{sec:intro}

Esta etapa do processo de seleção consiste em implementar uma aplicação capaz
de realizar algumas operações comuns à segurança da computação. O objetivo do
desafio não é avaliar o seu grau de conhecimento, mas sim como você se comporta
ao se deparar com um desafio.

Aqui serão considerados alguns fatores: (i) quanto você se dedicou para
resolver o desafio; (ii) quanto do desafio foi resolvido; (iii) qual
o comportamento que teve quando surgiu dúvidas; (iv) sua organização do código;
(v) entre outros fatores.

As operações que a aplicação deverá conter são as seguintes:

\begin{itemize}
  \item obter o resumo criptográfico de um documento;
  \item gerar um par de chaves assimétricas;
  \item gerar certificados digitais;
  \item gerar repositórios de chaves;
  \item gerar uma assinatura digital;
  \item verificar uma assinatura digital.
\end{itemize}

No capítulo seguinte, serão apresentados alguns conceitos básicos, necessários
para o bom entendimento da implementação do desafio.

Desde já, agradecemos o esforço do autor original deste desafio e documento,
Lucas Ferraro.

\section{Conceitos básicos de criptografia}\label{sec:basiccrypto}

Segundo Menezes \textit{et al.}~\cite{Menezes:book:1996}, criptografia
é o estudo de técnicas matemáticas relacionadas a aspectos da segurança da
informação, tais como confidenciabilidade, integridade, autenticação
e certificação. De todos os objetivos da segurança da informação, os quatro
acima formam uma base, onde os demais são derivados deles.

\begin{itemize}
  \item \textbf{Confidencialidade} é um serviço usado para manter o conteúdo da
      informação para todos que tenham autorização para tal. O segredo
        é sinônimo de confidencialidade e privacidade. Existem várias
        abordagens para proporcionar confidencialidade, que vão desde
        a proteção física até algoritmos matemáticos que tornam os dados
        ininteligíveis.

  \item \textbf{Integridade} é um serviço que trata da alteração não autorizada
      de dados. Para assegurar a integridade dos dados, é preciso conseguir
        detectar a manipulação dos dados por terceiros não autorizados. Essa
        manipulação se refere a ações como inserção, exclusão e substituição.

  \item \textbf{Autenticação} é um serviço relacionado à identificação. Essa
      função se aplica a ambas as entidades e a própria informação. Duas partes
        que começam uma comunicação devem se identificar uma à outra.

  \item \textbf{Não-repúdio} é um serviço que evita uma entidade de negar
      autorizações ou ações anteriores. Quando surgir alguma disputa em que uma
        entidade não admita que autorizou que certa ação fosse tomada,
        é necessária uma forma de resolver o problema, então uma terceira
        entidade confiável é convocada para resolver a disputa.
\end{itemize}

A criptografia é caracterizada por três diferentes dimensões. A primeira
é a operação usada para transformar o texto plano, ou seja, o texto original,
em texto cifrado. O segundo é a quantidade de chaves usadas e o terceiro
é a forma que o texto plano é processado. Para esse desafio, o ponto mais
importante é o segundo, pois diferencia a criptografia simétrica da
assimétrica, ou de chaves públicas. A seguir será apresentado um breve conceito
de criptografia simétrica e uma forma de distribuição de chaves confiável para
comunicação segura entre duas entidades distintas. Após, será visto o conceito
de criptografia assimétrica e suas aplicações.

\subsection{Criptografia simétrica}\label{subsec:symmetric}

Segundo Paar \textit{et al.}~\cite{Paar:book:2009}, a melhor maneira de definir
a criptografia simétrica de forma bem simples é através do exemplo abaixo.

Tem-se dois usuários, Alice e Bob, que querem se comunicar através de um canal
inseguro. O termo canal pode soar um pouco abstrato, mas é apenas uma forma
geral para dizer que eles querem se comunicar, por exemplo, pela
Internet. O verdadeiro problema começa quando um terceiro usuário com más
intenções aparece na situação, Oscar, que deseja roubar informações da
comunicação entre Alice e Bob. É claro que existem diversas maneiras de Alice
e Bob se comunicarem sem serem ouvidos, entretanto, eles estão em seus
escritórios e desejam enviar um ao outro documentos secretos e ninguém, além
deles, pode saber do que se trata o conteúdo, como na Figura~\ref{fig:1}.

\begin{figure}[!ht]
  \centering
  \begin{tikzpicture}
    \begin{scope}[every node/.style={draw, thick}]
      \node (A) at (0, 0) {Alice};
      \node (B) at (8, 0) {Bob};
      \node (O) at (4, 2) {Oscar};
      \node[fill=gray!40] (C) at (4, 0) {canal inseguro};
    \end{scope}
    \begin{scope}[every node/.style={fill=white, circle}]
      \path [->] (A) edge node {$x$} (C);
      \path [->] (C) edge node {$x$} (B);
      \path [->] (C) edge node {$x$} (O);
    \end{scope}
  \end{tikzpicture}
  \caption{Comunicação insegura.}\label{fig:1}
\end{figure}

Uma das soluções para a situação de Alice e Bob é criar uma chave secreta
e trocar essa chave por um canal seguro. Assim, usando um algoritmo de
criptografia simétrica, Alice poderia cifrar os dados originais com essa chave
secreta e enviar os dados cifrados para Bob. Este, conhecendo o algoritmo usado
para cifrar, poderia usar a mesma chave já combinada anteriormente, e decifrar
o dado para assim ter o texto original de forma segura. Dessa forma Oscar pode
até interceptar a mensagem, porém não poderá decifrá-la, pois não tem
conhecimento da chave secreta e não teria como decifrar o texto, como na
Figura~\ref{fig:2}.

Se Alice e Bob não tivessem como combinar uma chave secreta por um meio seguro,
essa solução não poderia ser utilizada. Em razão disso, começaram a ser
elaborados protocolos para troca de chaves, sendo que um dos primeiros foi
o protocolo de Diffie--Hellman, que mostra como usar a criptografia simétrica
com duas chaves diferentes. Também a partir desse problema e da ideia de
Diffie e Hellman, surgiu a ideia da criptografia assimétrica e da distribuição
de chaves por uma entidade confiável de forma hierárquica.

\begin{figure}[!ht]
  \centering
  \begin{tikzpicture}
    \begin{scope}[every node/.style={draw, thick}]
      \node (A) at (0, 0) {Alice};
      \node (E) at (2.5, 0) {cifra};
      \node (D) at (9.5, 0) {decifra};
      \node (B) at (12, 0) {Bob};
      \node (O) at (6, 2) {Oscar};
      \node[fill=gray!40] (C) at (6, 0) {canal inseguro};
      \node (S) at (6, -1) {canal seguro};
    \end{scope}
    \begin{scope}[every node/.style={fill=white, circle}]
      \path [->] (A) edge node {$x$} (E);
      \path [->] (E) edge node {$y$} (C);
      \path [->] (C) edge node {$y$} (D);
      \path [->] (D) edge node {$x$} (B);
      \path [->] (C) edge node {$y$} (O);
      \path [->, line width=1pt] (S) edge [bend left=25] node {$k$} (E);
      \path [->, line width=1pt] (S) edge [bend right=25] node {$k$} (D);
    \end{scope}
  \end{tikzpicture}
  \caption{Criptografia simétrica.}\label{fig:2}
\end{figure}

\subsection{Criptografia assimétrica}\label{subsec:asymmetric}

Segundo Stallings~\cite{Stallings:book:2016}, o conceito de criptografia
assimétrica, ou criptografia de chaves públicas, envolve uma tentativa de
resolver alguns dos problemas mais difíceis associados à criptografia
simétrica. Um deles é a distribuição de chaves, outro a falta de proteção na
comunicação entre duas entidades.

Na criptografia simétrica uma chave precisa ser estabelecida entre as duas
entidades por um canal seguro, como foi demonstrado na Figura~\ref{fig:2}, onde
apenas com a existência do canal seguro seria possível ocorrer uma comunicação
segura. Mesmo assim, se o problema dessa distribuição de chaves fosse
resolvido, o número de chaves existentes iria ser imenso, pois deveria existir
uma chave diferente para cada entidade e seus pares comunicantes. Ainda existe
outro fator a ser considerado, o não-repúdio.

Para resolver esses problemas, Diffie e Hellman~\cite{Diffie:article:1976:sep}
e Merkle~\cite{Merkle:phd:1979:jun} tiveram propostas revolucionárias, baseadas
na seguinte ideia: a chave usada para cifrar a mensagem não necessita ser
secreta. A parte essencial é que o receptor da mensagem pode decifrar apenas
com uma chave privada. Para concretizar isso, o receptor teria de ter publicado
uma chave anteriormente para que o emissor pudesse usá-la para cifrar
a mensagem. Assim, o receptor teria duas chaves, ou um par de chaves, onde uma
é a chave pública e a outra, é a privada~\cite{Paar:book:2009}. Ainda de acordo
com Stallings, o algoritmo de criptografia assimétrica traz uma importante
característica, a de ser computacionalmente inviável determinar a chave privada
pela chave pública relacionada.

Stallings define que um esquema de criptografia assimétrica é baseado em seis
ingredientes, como visto na Figura~\ref{fig:3}.

\begin{figure}[!ht]
  \centering
  \begin{tikzpicture}[scale=0.9]
    \begin{scope}[every node/.style={draw, thick}]
      \node[draw=none] (T1) at (-2, 0.75) {texto plano};
      \node[draw=none] (T2) at (12, 0.75) {texto plano};
      \node (P) at (-2, 0) {mensagem};
      \node (E) at (1.5, 0) {algoritmo para cifrar};
      \node (D) at (8.5, 0) {algoritmo para decifrar};
      \node (B) at (12, 0) {mensagem};
      \node (C) at (1.5, 4) {Conjunto de chaves públicas conhecidas};
      \node (Pk) at (1.5, 2) {Chave pública de A};
      \node (Sk) at (8.5, 2) {Chave privada de A};
      \node[draw=none] (S) at (5, -1) {Sistema de criptografia
        assimétrica para mensagem secreta};
    \end{scope}
    \begin{scope}[every node/.style={fill=white}]
      \path [->] (P) edge (E);
      \path [->] (C) edge (Pk);
      \path [->] (E) edge node {texto cifrado} (D);
      \path [->] (Pk) edge (E);
      \path [->] (Sk) edge (D);
      \path [->] (D) edge (B);
    \end{scope}

    \begin{scope}[every node/.style={draw, thick}]
      \node[draw=none] (T1) at (-2, -5.75) {texto plano};
      \node[draw=none] (T2) at (12, -5.75) {texto plano};
      \node (P) at (-2, -6.5) {mensagem};
      \node (E) at (1.5, -6.5) {algoritmo para cifrar};
      \node (D) at (8.5, -6.5) {algoritmo para decifrar};
      \node (B) at (12, -6.5) {mensagem};
      \node (C) at (8.5, -2.5) {Conjunto de chaves públicas conhecidas};
      \node (Sk) at (1.5, -4.5) {Chave privada de A};
      \node (Pk) at (8.5, -4.5) {Chave pública de A};
      \node[draw=none] (S) at (5, -7.5) {Sistema de criptografia
        assimétrica para autenticação};
    \end{scope}
    \begin{scope}[every node/.style={fill=white}]
      \path [->] (P) edge (E);
      \path [->] (C) edge (Pk);
      \path [->] (E) edge node {texto cifrado} (D);
      \path [->] (Sk) edge (E);
      \path [->] (Pk) edge (D);
      \path [->] (D) edge (B);
    \end{scope}
  \end{tikzpicture}
  \caption{Criptografia assimétrica.}\label{fig:3}
\end{figure}

\begin{itemize}
  \item \textbf{texto plano} é a mensagem em sua forma original e legível,
      também pode ser um dado qualquer. Essa é a entrada para o algoritmo
        responsável por cifrar.

  \item \textbf{algoritmo para cifrar} faz várias transformações com o texto
      plano para deixá-lo ilegível.

  \item \textbf{par de chaves} são dois ingredientes da criptografia
      assimétrica, a chave pública e a chave privada, sendo que uma é usada
        para cifrar e a outra para decifrar. O algoritmo para cifrar depende da
        chave usada como entrada.

  \item \textbf{texto cifrado} é a saída do algoritmo usado para cifrar, aqui
      o texto é totalmente ilegível. Esse texto, ou dado, é totalmente
        dependente do texto plano e da chave usada para cifrar.

  \item \textbf{algoritmo para decifrar} aceita um texto cifrado e sua chave
      correspondente para produzir o texto plano.
\end{itemize}

Existem ainda passos essenciais para esse sistema, que Stallings reporta como
segue:

\begin{enumerate}
  \item Cada usuário gera um par de chaves para cifrar e decifrar mensagens.

  \item Cada usuário põe uma das chaves (note que pode ser qualquer uma delas,
      mas apenas uma) em um repositório ou qualquer local acessível ao
        público. Essa é chamada chave pública. A chave restante é a privada,
        a qual ninguém deve ter acesso, como foi visto na
        Figura~\ref{fig:3}. Cada usuário mantém uma coleção de chaves públicas
        obtida dos outros usuários.

  \item Se uma entidade $A$ desejar mandar uma mensagem confidencial para
      entidade $B$, então $A$ precisa usar a chave pública de $B$ para cifrar
        a mensagem.

  \item Quando $B$ recebe a mensagem, ele a decifra usando sua chave
      privada. Nenhum outro usuário pode decifrar a mensagem porque apenas $B$
        conhece sua chave privada.
\end{enumerate}

Com esse procedimento, todos os participantes têm acesso a todas as chaves
públicas, e as chaves privadas são geradas localmente por cada participante
e não podem ser distribuídas. Enquanto a chave privada de um usuário está
intacta, comunicações envolvendo esse usuário são seguras. A qualquer momento,
algum dos usuários pode trocar seu par de chaves e substituir a chave pública
antiga pela nova.

Stallings ainda cita que a criptografia assimétrica tem três classes de uso:

\begin{itemize}
  \item \textbf{cifrar e decifrar}: o emissor cifra a mensagem com a chave
      pública do destinatário;

  \item \textbf{assinatura digital}: o emissor assina a mensagem com sua chave
      privada, visto com mais detalhes na Subseção~\ref{subsec:digitalsig};

  \item \textbf{distribuição de chaves}: existem vários protocolos e serviços
      de trocas de chaves, mas o que será visto nesse desafio será uma
        estrutura hierárquica para distribuição e controle das chaves, que será
        detalhada na Subseção~\ref{subsec:pki}.
\end{itemize}

\subsection{Resumo criptográfico}\label{subsec:hash}

Segundo Ferguson \textit{et al.}~\cite{Ferguson:book:2011}, resumo
criptográfico, também conhecido por função de resumo criptográfico, é uma
função que recebe uma entrada arbitrária de bits, e transforma em uma saída de
tamanho fixo. Um uso típico da função de resumo criptográfico é na assinatura
digital, onde, ao invés de assinar a mensagem $m$, que pode ter um tamanho de
alguns milhões de \textit{bytes}, pode-se aplicar primeiro uma função de resumo
criptográfico e deixar a mensagem $m$ com um tamanho fixo e muito menor, por
exemplo, com 256 bits. Essa aplicação $H(m)$, pode ser chamada resumo
criptográfico de $m$ e oferece um desempenho muito maior ao processo de
assinatura digital.

Uma função de resumo criptográfico, ou simplesmente, uma função \textit{hash}
$H$ precisa ter as seguintes propriedades~\cite{Stallings:book:2016}:

\begin{itemize}
  \item $H$ tem que ser aplicável a um bloco de dados $x$ de qualquer tamanho;

  \item $H$ precisa produzir uma saída de tamanho fixo;

  \item $H(x)$ é relativamente fácil de ser computada para qualquer $x$,
      fazendo com que as implementações de \textit{hardware}
        e \textit{software} sejam práticas;

  \item Para qualquer valor de resumo criptográfico $h$, é impossível descobrir
      o bloco de dados $x$ que o gerou;

  \item Para qualquer $x$, deverá ser computacionalmente inviável encontrar um
      bloco $y \neq x$ tal que $H(y) = H(x)$.
\end{itemize}

O resumo criptográfico, por ser uma forma canônica de representar um bloco de
dados de tamanho arbitrário, pode ser usado como uma forma de garantir
a integridade de uma mensagem, pois é possível enviar a mensagem por completo
a um destinatário e com ela enviar um resumo criptográfico dessa
mensagem. Assim, antes de ler a mensagem recebida, o destinatário, conhecendo
a função de resumo criptográfico usada, poderia aplicar essa função à mensagem
e assim conferir se o resumo criptográfico recém-gerado é o mesmo que
o enviado.

Agora, ele poderia ler a mensagem e ter certeza sobre sua integridade. Essa
forma de uso do resumo criptográfico acrescentada à criptografia assimétrica
é o que fundamenta a assinatura digital. Dessa forma o emissor usa sua chave
privada para cifrar o resumo criptográfico da mensagem a ser enviada
e o receptor confere a mensagem enviada usando a chave pública do emissor para
recriar o resumo criptográfico, e assim ter a certeza que a mensagem
é autêntica e íntegra.

\subsection{Assinatura digital}\label{subsec:digitalsig}

Em situações em que não existe confiança mútua entre o emissor e o receptor de
mensagens, algo além da autenticação é necessário. A maneira mais atrativa de
solucionar isso é com assinatura digital, análoga à assinatura
manuscrita~\cite{Stallings:book:2016}. Para tanto, a assinatura digital deve
conter as seguintes propriedades:

\begin{itemize}
  \item deve verificar o autor da assinatura;

  \item tem que autenticar o conteúdo no momento da assinatura;

  \item precisa ser verificável por terceiros, para resolver disputas.
\end{itemize}

Com isso, a assinatura digital inclui a função de autenticação. Stallings diz
que com essas propriedades é possível formular os seguintes requisitos para
a assinatura digital:

\begin{itemize}
  \item a assinatura precisa ser um padrão de bits, que depende do conteúdo que
      está sendo assinado;

  \item a assinatura precisa usar alguma informação única para o emissor, para
      prevenir a falsificação do conteúdo e da autoria;

  \item precisa ser relativamente fácil gerar uma assinatura digital;

  \item precisa ser relativamente fácil reconhecer e verificar uma assinatura
      digital;

  \item precisa ser computacionalmente inviável forjar a assinatura digital,
      tanto construindo uma nova mensagem para uma assinatura já existente,
        quanto construindo uma assinatura fraudulenta para uma mensagem
        qualquer;

  \item precisa ser prático guardar uma cópia da assinatura digital.
\end{itemize}

Um método básico para criar uma assinatura digital seria primeiro gerar
o resumo criptográfico dos dados a serem assinados e depois o assinante teria
de aplicar sua chave privada a esse resumo. Dessa forma seria possível
verificar a autenticidade da assinatura e a integridade dos dados
assinados. Para tal verificação, basta aplicar a mesma função de resumo
criptográfico $H$ aos dados assinados e obter $h'$, que deve ser igual ao
resumo criptográfico cifrado pelo assinante.

Para decifrar tal resumo, é necessário conhecer a chave pública do assinante
e o algoritmo usado para cifrar. Feito isso, o resultado deve ser o resumo
criptográfico $h$ dos dados assinados. Caso $h = h'$ está comprovada
a integridade e autenticidade da assinatura digital; caso contrário, essa
assinatura foi falsificada e não deve ser confiável. As assinaturas digitais
têm muitas aplicações na segurança da informação, incluindo autenticação,
integridade de dados e não-repúdio.

Uma das mais importantes aplicações das assinaturas digitais é a certificação
de chaves públicas em grandes grupos. Certificação é um meio para uma terceira
parte confiável ligar a identidade de um usuário à sua chave pública. Assim,
algum tempo depois, outras entidades podem verificar a autenticidade da chave
pública sem a necessidade de uma terceira parte~\cite{Menezes:book:1996}. Isso
será abordado com mais detalhes na Subseção~\ref{subsec:digitalcert}.

\subsection{Certificado digital}\label{subsec:digitalcert}

Segundo Menezes \textit{et al.}, um certificado digital é uma estrutura de
dados que consiste em uma parte de dados e uma parte de assinatura. A parte de
dados contém um texto legível, incluindo, no mínimo, a chave pública e um texto
identificando a entidade associada a essa chave pública. Nota-se que essa
entidade pode ser uma pessoa. A parte da assinatura consiste em uma assinatura
digital feita por uma Autoridade Certificadora (AC, explicada na
Subseção~\ref{subsubsec:ac}) sobre a parte de dados da entidade requerente,
assim unindo a identidade da entidade com sua respectiva chave pública.

Para gerar um certificado digital, a princípio é necessário seguir os seguintes
passos:

\begin{enumerate}
  \item escolher qual será a AC emissora do certificado;

  \item solicitar a essa AC a emissão de um certificado digital, informando os
      dados do solicitante, tal como seu nome e sobrenome (os dados aqui
        dependem muito da política usada para a certificação), e esses formarão
        a parte de dados do certificado digital;

  \item em alguns casos, a AC pode solicitar que o usuário solicitante vá até
      uma autoridade de registro, a qual tem a incumbência de validar os dados
        informados na solicitação do certificado;

  \item caso todos os dados do solicitante estiverem de acordo, a autoridade de
      registro envia a requisição para a AC, que emite o certificado do
        solicitante.
\end{enumerate}

A verificação de um certificado digital é feita da mesma maneira que
a verificação de uma assinatura digital. Nota-se que a chave pública usada
nesse processo será a chave pública da AC emissora do certificado digital. Essa
verificação citada pode implicar em um resultado negativo, caso o certificado
tenha sido emitido pela AC, mas seja invalidado por alguma razão. Um dessas
razões é o certificado ter expirado, ou seja, ter chegado ao fim do tempo
previsto de validade. Outra maneira do certificado estar inválido é caso ele
tenha sido revogado. Segundo Stallings, existem três possibilidades para
a revogação do certificado, que são:

\begin{itemize}
  \item o usuário avisa que sua chave privada foi comprometida;

  \item o usuário não é mais certificado pela AC emissora;

  \item o certificado da AC emissora foi comprometido.
\end{itemize}

Para manter um registro que algum certificado foi revogado, as ACs guardam essa
informação de revogação em uma estrutura chamada lista de certificados
revogados (LCR), que está descrita na Subseção~\ref{subsubsec:crl}.

\subsection{Infraestrutura de chaves públicas}\label{subsec:pki}

Citando Ferguson \textit{et al.}, uma infraestrutura de chaves públicas (ICP)
é uma infraestrutura que permite a um usuário reconhecer a quem uma chave
pública pertence.

A estrutura clássica de uma ICP é baseada em hierarquia de autoridades,
parecida com um grafo no formato de árvore. A ideia aqui é ter um ponto de
confiança, que normalmente é uma autoridade certificadora, que por ser de
confiança e ser o nó inicial dessa estrutura, é chamada AC-Raiz. Além da
AC-Raiz, a autoridade máxima dessa estrutura, existem as ACs intermediárias
e ACs finais, essas últimas responsáveis pela emissão dos certificados digitais
dos usuários finais dessa infraestrutura.

Quando alguma entidade deseja validar um certificado digital de outra entidade,
ela precisa conhecer a AC emissora do certificado de tal entidade. Conhecido
isso, a entidade pode verificar a assinatura aposta no certificado usando
a chave pública da autoridade emissora. Esse mesmo processo tem que ser
realizado para a AC emissora e a AC superior a essa emissora, para verificar se
a tal AC emissora é válida.

Isso se repete até que a AC-Raiz seja alcançada, e quando isso ocorre,
a verificação termina, visto que a chave pública da AC-Raiz é publicada por
outros meios; por exemplo, a AC-Raiz da Infraestrutura de Chaves Públicas
Brasileira (ICP-Brasil) é publicada no Diário Oficial da União. Nota-se que
esta é uma forma simplificada para esse processo de validação, pois o mesmo
envolve outros passos que vão além da verificação das assinaturas nos
certificados.

\subsubsection{Autoridade certificadora}\label{subsubsec:ac}

Segundo Menezes \textit{et al.}, a autoridade certificadora (AC) é uma entidade
confiável cuja assinatura sobre o certificado digital comprova que a chave
pública pertence à entidade identificada no certificado.

\subsubsection{Lista de certificados revogados}\label{subsubsec:crl}

As listas de certificados revogados são estruturas em forma de lista emitidas
por ACs, de todos os níveis, onde são mantidas informações sobre qual
certificado emitido, em algum momento, está revogado. Essas listas são
publicadas periodicamente e assinadas pela própria AC emissora.

De acordo com Stallings, a única forma que as ACs tem para identificar um
certificado presente na LCR é o \textit{serial number} do certificado, um
identificador numérico. Esse é emitido pela AC junto ao certificado de um
usuário, e as buscas por certificados revogados dentro da LCR são feitas
através desse identificador numérico.

\section{Etapas do desafio}\label{sec:steps}

Dados os conceitos básicos, então, é chegada a hora de implementá-los. Com
o intuito de auxiliar no desenvolvimento das tarefas do desafio foi elaborado
um fluxo aconselhável da ordem de implementação. Note que não é obrigatório
segui-lo, mas pode ser mais fácil compreender o que está sendo feito dessa
forma.

Arquivos de interesse:

\begin{itemize}
  \item \texttt{hiring-0.2.3-src.tar.gz}, entregue junto a este
      documento. Contém uma estrutura para auxiliar no desenvolvimento, vista
        em detalhes na Seção~\ref{sec:impl};

  \item \texttt{artefatos/textos/textoPlano.txt} é um arquivo que será usado em
      várias etapas, encontrado no pacote acima. \textbf{Deverá conter seu nome
        e matrícula da graduação}.
\end{itemize}

\subsection{Primeira etapa: obter o resumo criptográfico de um
documento}\label{subsec:step1}

Basta obter o resumo criptográfico do documento \texttt{textoPlano.txt}.

Os pontos a serem verificados para essa etapa ser considerada concluída são os
seguintes:

\begin{itemize}
  \item obter o resumo criptográfico do documento, especificado na descrição
      dessa etapa, usando o algoritmo de resumo criptográfico SHA-256;

  \item armazenar em disco o arquivo contendo o resultado do resumo
      criptográfico, em formato hexadecimal.
\end{itemize}

\subsection{Segunda etapa: gerar chaves assimétricas}\label{subsec:step2}

A partir dessa etapa, tudo que será feito envolve criptografia
assimétrica. A tarefa aqui é parecida com a etapa anterior, pois se refere
apenas a criar e armazenar chaves, mas nesse caso será usado apenas um
algoritmo de criptografia assimétrica, o ECDSA\@.

Os pontos a serem verificados para essa etapa ser considerada concluída são os
seguintes:

\begin{itemize}
  \item gerar um par de chaves usando o algoritmo ECDSA com o tamanho de 256
      bits;

  \item gerar outro par de chaves, mas com o tamanho de 521 bits. Note que esse
      par de chaves será para a AC-Raiz;

  \item armazenar em disco os pares de chaves em formato PEM.
\end{itemize}

\subsection{Terceira etapa: gerar certificados digitais}\label{subsec:step3}

Aqui você terá que gerar dois certificados digitais. A identidade ligada a um
dos certificados digitais deverá ser a sua. A entidade emissora do seu
certificado será a AC-Raiz, cuja chave privada já foi previamente
gerada. Também deverá ser feito o certificado digital para a AC-Raiz, que
deverá ser autoassinado.

Os pontos a serem verificados para essa etapa ser considerada concluída são os
seguintes:

\begin{itemize}
  \item emitir um certificado digital autoassinado no formato X.509 para
      a AC-Raiz;

  \item emitir um certificado digital no formato X.509, assinado pela
      AC-Raiz. O certificado deve ter as seguintes características:

  \begin{itemize}
    \item \texttt{Subject} deverá ser o seu nome;

    \item \texttt{SerialNumber} deverá ser o número da sua matrícula;

    \item \texttt{Issuer} deverá ser a AC-Raiz.
  \end{itemize}

  \item armazenar em disco os certificados emitidos em formato PEM;

  \item as chaves utilizadas nessa etapa deverão ser as mesmas já geradas.
\end{itemize}

\subsection{Quarta etapa: gerar repositório de chaves
seguro}\label{subsec:step4}

Essa etapa tem como finalidade gerar um repositório seguro de chaves
assimétricas. Esse repositório deverá ser no formato PKCS\#12. Note que esse
repositório é basicamente uma tabela de espalhamento com pequenas mudanças. Por
exemplo, sua estrutura seria algo como \texttt{<Alias, <Certificado, Chave
Privada>>}, onde o \textit{alias} é um nome amigável dado a uma entrada da
estrutura, o certificado e chave privada devem ser correspondentes à mesma
identidade. O \textit{alias} serve como elemento de busca dessa
identidade. O PKCS\#12 ainda conta com uma senha, que serve para cifrar
a estrutura (isso é feito de modo automático).

Os pontos a serem verificados para essa etapa ser considerada concluída são os
seguintes:

\begin{itemize}
  \item gerar um repositório para o seu certificado/chave privada com senha
      e \textit{alias} conforme as constantes fornecidas;

  \item gerar um repositório para o certificado/chave privada da AC-Raiz com
      senha e \textit{alias} conforme as constantes fornecidas.
\end{itemize}

\subsection{Quinta etapa: gerar uma assinatura digital}\label{subsec:step5}

Essa etapa é um pouco mais complexa, pois será necessário que implemente um
método para gerar assinaturas digitais. O padrão de assinatura digital adotado
será o \textit{Cryptographic Message Syntax} (CMS). Esse padrão usa o formato
ASN.1, uma notação em binário, assim não será possível ler o resultado obtido
sem o auxílio de alguma ferramenta. Caso tenha interesse em ver a estrutura da
assinatura gerada, então, o uso da ferramenta
\texttt{\href{https://www.cs.auckland.ac.nz/~pgut001/dumpasn1.c}{dumpasn1}}
é recomendado.

Os pontos a serem verificados para essa etapa ser considerada concluída são os
seguintes:

\begin{itemize}
  \item gerar uma assinatura digital usando o algoritmo de resumo criptográfico
      SHA-256 e o algoritmo de assinatura digital ECDSA\@;

  \item o assinante será você. Então, use o PKCS\#12 recém-gerado para seu
      certificado e chave privada;

  \item assinar o documento \texttt{textoPlano.txt}, onde a assinatura deverá
      ser do tipo ``anexada'', ou seja, o documento estará embutido no arquivo
        de assinatura;

  \item gravar a assinatura em disco.
\end{itemize}

\subsection{Sexta etapa: verificar uma assinatura digital}\label{subsec:step6}

Por último, será necessário verificar a integridade da assinatura
recém-gerada. Note que o processo de validação de uma assinatura digital pode
ser muito complexo, mas aqui o desafio será simples. Para verificar
a assinatura será necessário apenas decifrar o valor da assinatura (resultante
do processo de cifra do resumo criptográfico do arquivo \texttt{textoPlano.txt}
com as informações da estrutura da assinatura) e comparar esse valor com
o valor do resumo criptográfico do arquivo assinado. Como dito na
fundamentação, para assinar é usada a chave privada, e para decifrar
(verificar) é usada a chave pública.

Os pontos a serem verificados para essa etapa ser considerada concluída são os
seguintes:

\begin{itemize}
  \item verificar a assinatura gerada na etapa anterior, conforme o
      processo descrito, e apresentar esse resultado.
\end{itemize}

\section{Implementação}\label{sec:impl}

A implementação deverá seguir algumas regras consideradas na avaliação do
desafio. O desafio estará comprometido se as regras não forem
respeitadas. Essas regras são as seguintes:

\begin{itemize}
  \item o desafio, codificado em Java, é apresentado como um projeto
      Maven. Sugere-se que a IDE utilizada seja o IntelliJ IDEA\@. A estrutura
        do projeto \textbf{não deve} divergir significativamente da sugerida;

  \item junto ao código, deverá ser entregue um \textbf{relatório} elaborado
      a partir das atividades desenvolvidas, ligando a fundamentação
        apresentada anteriormente com o código escrito. Note que as soluções
        para o desafio são conhecidas, mas este documento avaliará seu
        compreendimento destes conceitos. O documento deverá estar \textbf{em
        formato PDF, na mesma pasta deste};

  \item a entrega deverá ser feita através da plataforma onde este desafio foi
      fornecido. O arquivo entregue deverá \textbf{obrigatoriamente} ser
        o resultado do comando \texttt{mvn package};

  \item o prazo de entrega é em duas semanas após o recebimento deste desafio;

  \item se detectado plágio, o candidato estará \textbf{expulso} do processo de
      seleção. Caso alguma referência for utilizada, é necessário citá-la como
        parte da documentação do código.
\end{itemize}

Para ajudar a guiar no desenvolvimento do desafio foi elaborada uma estrutura
contendo as possíveis classes, pacotes e locais para guardar os artefatos
gerados durante a implementação. Nessa estrutura são dadas algumas dicas de
como se deve proceder para a boa execução do desafio.

Durante a implementação, se houver qualquer dúvida, problema, ou outra coisa
que o impeça de continuar o desafio, entre em contato da forma que achar
melhor. Lembre-se que esperamos que você traga perguntas específicas e que
demonstre que ao menos tentou implementar as etapas propostas. Quanto mais
genérica for a pergunta, mais genérica será a resposta. Geralmente as respostas
virão na forma de referências a algo que responda sua questão.

Os contatos são aqueles dados no início desse documento e/ou na plataforma
específica para o processo seletivo. E caso ache realmente necessário conversar
pessoalmente, então apareça no Laboratório ou use uma sala de
videoconferência. Apenas envie um e-mail confirmando a disponibilidade. Boa
sorte!

\bibliography{\jobname}
\bibliographystyle{alpha}

\end{document}
